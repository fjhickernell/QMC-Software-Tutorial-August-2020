%  LaTeX template for abstract submission for MCQMC 2020
% 
%  If this paper is for a special (invited) session, put the name of session organizer 
%  (first name, family name), and the session title.
%  For plenary talks, put ``Plenary Talk'' for the session title, and Bruno Tuffin for the organizer.
% If the paper is for a symposium, proceed similarly with command \insymposium 
% Comment or remove the unnecessary line.
% If neither in a minisymposium or special session, please comment the two lines. 
% Recall that it is just for the submission, it will not appear in the final program.
%
%\insession{Mike}{Giles}{Plenary Talk}
\insession{Mike}{Giles}{Tutorial}
%\insymposium{1stNameOrganizer}{LastNameOrganizer}{Symposium Title}


% First name and name of the speaker.
\speaker{Fred J.}{Hickernell}%
%  (put no space here)
% Title of the talk, capitalized.
\title{Quasi-Monte Carlo Software}

% For each author, give the first name, family name, affiliation, and email.
% Ideally, the affiliation and email should fit on a single line.  
% No need to put the full snail mailing address.  
%  One line per author 
\author{Sou-Cheng}{Choi}{Kamakura Corporation, USA and Department of Applied Mathematics, Illinois Institute of Technology, USA}{schoi32@iit.edu}
\author{Fred J.}{Hickernell}{Department of Applied Mathematics, Illinois Institute of Technology, USA}{hickernell@iit.edu}
\author{R.}{Jagadeeswaran}{Department of Applied Mathematics, Illinois Institute of Technology, USA}{jrathin1@hawk.iit.edu}
\author{Michael J.}{McCourt}{SigOpt, USA}{mccourt@sigopt.com}
\author{Aleksei}{Sorokin}{Department of Applied Mathematics, Illinois Institute of Technology, USA}{asorokin@hawk.iit.edu}


% Type your abstract here.
\abstract{
	Quasi-Monte Carlo (QMC) methods achieve substantial efficiency gains by replacing independent and identically distributed (IID) random points by low discrepancy (LD) points.  LD point generators and QMC algorithms are active research areas.  Practitioners are attracted to QMC by the promise of  efficiency gains.
	
	This tutorial highlights several readily available QMC software libraries in various languages.  We describe the components of a QMC calculation:  the LD point generators, problem specification, methods for speeding up the computation, and stopping criteria.  We argue that excellent QMC software requires the collaboration of a community---not only the efforts of individual research groups.  
	
	During this tutorial we provide hands-on experience with QMCPy \cite{QMCPy}, a library that draws on the work of several experts using Google Colaboratory:  \url{https://tinyurl.com/QMCPyTutorial}.  QMCPy grew out of discussions held at MCQMC 2018.  Minimal experience with QMC or Python is assumed.

	
	
%  If you have refererences, put them here in a format like below. 
%  This can be obtained using BiBTeX with the bib style plain.bst, uncomenting first the two next lines and replacing them by the generated .bbl file
% \bibliographystyle{plain} 
% \bibliography{MyBibFileName}
% 
%  Note that this bibilography must be placed inside the abstract.
\begin{thebibliography}{1}

\bibitem{QMCPy}
S.-C.\ T.\ Choi, F.\ J.\ Hickernell, R.\ Jagadeeswaran, M.\ J.\ McCourt  \& A.\ Sorokin,
\newblock QMCPy: A quasi-Monte Carlo Python Library \url{https://qmcsoftware.github.io/QMCSoftware/},
\newblock 2020.

\end{thebibliography}
}  % End of abstract.


