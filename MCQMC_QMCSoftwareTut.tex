%Talk given virtually for SIAM UQ 2020 March
\documentclass[11pt,compress,xcolor={usenames,dvipsnames},aspectratio=169]{beamer}
%\documentclass[xcolor={usenames,dvipsnames},aspectratio=169]{beamer} %slides and 
%notes
\usepackage{amsmath,
	amssymb,
	datetime,
	mathtools,
	bbm,
	%mathabx,
	array,
	booktabs,
	xspace,
	calc,
	colortbl,
	siunitx,
 	graphicx}
\usepackage[usenames]{xcolor}
\usepackage[giveninits=false,backend=biber,style=nature, maxcitenames =10, mincitenames=9]{biblatex}
\addbibresource{FJHown23.bib}
\addbibresource{FJH23.bib}
\usepackage{newpxtext}
\usepackage[euler-digits,euler-hat-accent]{eulervm}
\usepackage{media9}
\usepackage[autolinebreaks]{mcode}
\usepackage[tikz]{mdframed}


\usetheme{FJHSlimNoFoot169}
\setlength{\parskip}{2ex}
\setlength{\arraycolsep}{0.5ex}

\DeclareMathOperator{\sol}{SOL}
\DeclareMathOperator{\app}{APP}
\DeclareMathOperator{\alg}{ALG}
\DeclareMathOperator{\ACQ}{ACQ}
\DeclareMathOperator{\ERR}{ERR}
\DeclareMathOperator{\COST}{COST}
\DeclareMathOperator{\COMP}{COMP}
\newcommand{\dataN}{\bigl(\hf(\vk_i)\bigr)_{i=1}^n}
\newcommand{\dataNj}{\bigl(\hf(\vk_i)\bigr)_{i=1}^{n_j}}
\newcommand{\dataNjd}{\bigl(\hf(\vk_i)\bigr)_{i=1}^{n_{j^\dagger}}}
\newcommand{\ERRN}{\ERR\bigl(\dataN,n\bigr)}

\newcommand{\Sapp}{S_{\textup{app}}}
\newcommand{\LambdaStd}{\Lambda^{\textup{std}}}
\newcommand{\LambdaSer}{\Lambda^{\textup{ser}}}
\newcommand{\LambdaAll}{\Lambda^{\textup{all}}}
\newcommand{\oton}{1\!:\!n}
\newcommand{\talert}[1]{\alert{\text{#1}}}
\DeclareMathOperator{\init}{init}
\DeclareMathOperator{\GP}{\cg\cp}
\newcommand{\MLE}{\textup{EB}}
\newcommand{\mCtheta}{{\mathsf{C}_{\vtheta}}}
\newcommand{\mCInv}{\mathsf{C}^{-1}}

%\DeclareMathOperator{\app}{app}

\providecommand{\HickernellFJ}{H.\xspace}


\iffalse
The Successes and Challenges of Automatic Bayesian Cubature

The promise of Bayesian cubature is that with reasonable prior information (or assumptions) about the integrand, one can construct an optimal approximation to the corresponding (multidimensional) integral and simultaneously a credible interval.  Automatic Bayesian cubature uses increases the sample size until half-width of the credible interval is small enough.  We discuss how to choose appropriate covariance kernels, estimate their hyper-parameters, and construct the credible intervals, all with reasonable computational effort.  We also evaluate the performance of automatic Bayesian cubature on a variety of examples.


\fi

\renewcommand{\OffTitleLength}{-10ex}
\setlength{\FJHThankYouMessageOffset}{-8ex}
\title{Quasi-Monte Carlo Software}
\author[]{Fred J. Hickernell}
\institute{Department of Applied Mathematics \&
	Center for Interdisciplinary Scientific Computation \\  Illinois Institute of Technology \quad
	\href{mailto:hickernell@iit.edu}{\url{hickernell@iit.edu}} \quad
	\href{http://mypages.iit.edu/~hickernell}{\url{mypages.iit.edu/~hickernell}}}

\thanksnote{with Sou-Cheng Choi, Mike McCourt, Aleksei Sorokin \\ 
and the rest of the GAIL an QMCPy teams \\
	partially supported by SigOpt and NSF-DMS-1522687 \\[2ex]
	Thanks to the organizers, especially during these unusual times \\
	Slides at  \href{https://speakerdeck.com/fjhickernell/quasi-monte-carlo-software}{\nolinkurl{speakerdeck.com/fjhickernell/quasi-monte-carlo-software}}\\
	Google Colaboratory notebook at \href{https://tinyurl.com/QMCPyTutorial}{\nolinkurl{tinyurl.com/QMCPyTutorial}}\\
	Blog at \href{https://qmcpy.wordpress.com/}{\nolinkurl{qmcpy.wordpress.com/}}
	
}
\event{MCQMC 2020}
\date[]{August 2020}

\input FJHDef.tex


\newlength{\figwidth}
\setlength{\figwidth}{0.25\textwidth}

\newlength{\figwidthSmall}
\setlength{\figwidthSmall}{0.2\textwidth}

\newcommand{\financePict}{\href{http://i2.cdn.turner.com/money/dam/assets/130611131918-chicago-board-options-exchange-1024x576.jpg}{\includegraphics[width
		= 3cm]{ProgramsImages/130611131918-chicago-board-options-exchange-1024x576.jpg}}}
	
	\newcommand{\scoop}[1]{\parbox{#1}{\includegraphics[width=#1]{IceCreamScoop.eps}}\xspace}
	\newcommand{\smallscoop}{\scoop{1cm}}
	\newcommand{\medscoop}{\scoop{1.8cm}}
	\newcommand{\largescoop}{\scoop{3cm}}
	\newcommand{\ICcone}[1]{\parbox{#1}{\includegraphics[width=#1,angle=270]{MediumWaffleCone.eps}}\xspace}
	\newcommand{\medcone}{\ICcone{1.2cm}}
	\newcommand{\largercone}{\parbox{2.2cm}{\vspace*{-0.2cm}\includegraphics[width=1cm,angle=270]{MediumWaffleCone.eps}}\xspace}
	\newcommand{\largecone}{\ICcone{1.8cm}}
	\newcommand{\smallcone}{\parbox{1.1cm}{\includegraphics[width=0.5cm,angle=270]{MediumWaffleCone.eps}}\xspace}

	

\newcommand{\northeaststuff}[3]{
	\begin{tikzpicture}[remember picture, overlay]
	\node [shift={(-#1 cm,-#2 cm)}]  at (current page.north east){#3};
	\end{tikzpicture}}


\begin{document}
	\tikzstyle{every picture}+=[remember picture]
	\everymath{\displaystyle}

\frame{\titlepage}


\section{Background}

\begin{frame}{You have heard about QMC.  How do you try it?}
	
	\vspace{-5ex}
\begin{itemize}
\setlength{\itemsep}{0cm}
    \item<1-> \emph{QMC will give you 100 times the accuracy in the same amount of time as simple MC} \\
    
    \item<1-> \emph{Just replace your IID random numbers with low discrepancy points}\\
    \uncover<2->{\alert{Often, Sometimes}}
    
    \vspace{4ex}
    
    \item<3-> \emph{Where can I get accurate, efficient, easy to use QMC software to try for my problem?}\\
     \item<3-> \emph{How can I make my great software available for others?}\\
    \uncover<4->{\alert{Let's try to help}}
    
\end{itemize}
\end{frame}

\begin{frame}{Quasi-Monte Carlo (QMC) Uses Low Discrepancy (LD) Sequences}
\vspace{-3ex}
\begin{tabular}{>{\centering}p{0.47\textwidth}@{\quad}>{\centering}p{0.47\textwidth}}
%Independent \& Identically Distributed (IID) &
%Low Discrepancy (LD) \tabularnewline
\includegraphics[height=5cm]{ProgramsImages/IIDPoints.eps} &
\includegraphics[height=5cm]{ProgramsImages/SSobolPoints.eps}
\tabularnewline
$\vT_i$ are random &
$\vX_i$ may be deterministic or random 
\tabularnewline
$\vT_1, \vT_2 \cdots \alert{\IIDsim} F$ &
$\vX_1, \vX_2 \cdots \alert{\LDsim} F$ 
\tabularnewline
$\vT_i$ do not know about one another &
$\{\vX_i\}_{i=1}^n$ represent $F$ well
\tabularnewline
$F_{n}(\vt_1, \ldots, \vt_n) = F(\vt_1) \cdots F(\vt_n)$ &
$F_{\{\vX_i\}_{i=1}^n}(\vx) \approx F(\vx)$
\end{tabular}
\end{frame}

\begin{frame}{Where Is the Software?}
			\vspace{-3ex}
	
	\renewcommand{\arraystretch}{1.35}
	\begin{tabular}{>{\centering}m{0.47\textwidth}@{\qquad}>{\centering}m{0.47\textwidth}}
		\alert{LD Sequence Generators} & \alert{Multi-Level, Stopping Criteria, Applications}
		\tabularnewline \toprule
		\uncover<1>{\href{https://www.mathworks.com}{\alert{MATLAB Statistics Toolbox}}---\newline Sobol' and Halton} &
		\href{https://people.maths.ox.ac.uk/gilesm/mlmc/}{\alert{Mike Giles}}---Multi-Level (Quasi-)Monte Carlo  \uncover<1>{in C++, MATLAB, Python, and R}
		\tabularnewline
		\href{https://cran.r-project.org/web/packages/qrng/qrng.pdf}{\alert{Marius Hofert \& Christiane Lemieux }}---\texttt{qrng} \uncover<1>{R package,} Sobol' and Halton &
	  \href{http://gailgithub.github.io/GAIL_Dev/}{\alert{Guaranteed Automatic Integration Library (GAIL)}}---Stopping criteria  \uncover<1>{MATLAB}
		\tabularnewline
		\href{http://statweb.stanford.edu/~owen/code/}{\alert{Art Owen}}---randomized Halton\uncover<1>{ in R}&
		\tabularnewline
		\uncover<1>{\href{https://github.com/PieterjanRobbe/QMC.jl}{\alert{Pieterjan Robbe}---LD sequences in Julia}}
		\tabularnewline
		\href{https://pytorch.org/}{\alert{PyTorch}---Sobol'}
		\tabularnewline
		\multicolumn{2}{>{\centering}m{0.96\textwidth}}{\href{http://simul.iro.umontreal.ca}{\alert{Pierre L'Ecuyer}---Lattice Builder \uncover<1>{and  Stochastic Simulation in C/C++ and Java}}}
		\tabularnewline
		\multicolumn{2}{>{\centering}m{0.96\textwidth}}{\href{https://people.cs.kuleuven.be/~dirk.nuyens/}{\alert{Dirk Nuyens}}---Magic Point Shop \uncover<1>{and QMC4PDE in MATLAB, Python, and C++}}
\tabularnewline
		\multicolumn{2}{>{\centering}m{0.96\textwidth}}{\uncover<1>{\href{http://people.sc.fsu.edu/~jburkardt/}{\alert{John Burkhardt}}---variety in C++, Fortran, MATLAB, \& Python}}
\tabularnewline
		\multicolumn{2}{>{\centering}m{0.96\textwidth}}{\href{https://qmcsoftware.github.io/QMCSoftware/}{\alert{QMCPy}}---Python package \alert<2->{drawing on and connecting} the work of different groups}
\tabularnewline
	\end{tabular}

\renewcommand{\arraystretch}{1}
    
\end{frame}


\begin{frame}{Why Is LD Better than IID?}
	\vspace{-4ex}
	\begin{description}
		\setlength{\itemsep}{0.5cm}
		\item[Integration]  Arising in finance, uncertainty quantification, Bayesian inference, \ldots
		\begin{equation*}
		\mu = \Ex[f(\vX)] = \int_\cx f(\vx) \, \varrho(\vx) \, \dif \vx \approx \hmu = \frac 1n \sum_{i=1}^n f(\vX_i)
		\end{equation*}
		LD points give faster convergence than IID
		
		\item[Design of Computer Experiments] LD can be more space filling (even than Latin hypercube sampling) for use in constructing surrogate models

		\item[Global Optimization]  LD points can be more space filling and find good starting points for other methods.
		
		\end{description}
	
    
\end{frame}

\begin{frame}{What Software Components Do We Need?}
	
	\vspace{-6ex}
	
	\[
	 \uncover<2->{\mu =  \int_{\ct} \alert<3>{g(\vt)} \, \alert<2>{\lambda(\vt) \, \dif \vt} = \cdots = } 	\underbrace{\Ex[f(\vX)]}_{\text{expectation}} = \underbrace{\int_\cx \alert<3>{f(\vx)} \, \varrho(\vx) \, \dif \vx}_{\text{integration}} \approx  \frac 1n \sum_{i=1}^{\alert<4>{n}} f(\alert<1>{\vX_i}) =: \hmu
	\]
	
		\vspace{-3ex}
	
	\begin{description}[<+->]
				\setlength{\itemsep}{0.5cm}
		
		\item[LD Generator] producing $\{\vX_1, \vX_2, \dots \}$ that mimics the distribution with PDF $\varrho$, e.g., uniform
		
		\item[True Measure] that defines the original integral, e.g., Lebesgue
		
		\item[Integrand] $g$, which defines the original integral, plus the transformed version, $f$, to fit the LD generator
		
		\item[Stopping Criterion] that determines how large $n$ needs to be, e.g., to ensure that $\abs{\mu - \hmu} \le \varepsilon$
	\end{description}
\end{frame}

\section{QMCPy on Google Colaboratory}
\begin{frame}{QMCPy in a Jupyter Notebooks on Google Colaboratory}
	
	\alert{Prerequisites}---No Python, Jupyter, or QMC knowledge assumed
	
	\alert{Goals}
	
		\vspace{-3ex}
		\begin{itemize}
		\item Show you how QMC software works
		\item Interest you in using/contributing
	\end{itemize}
	
	\alert{Directions}
	
	\vspace{-3ex}
	\begin{itemize}
		\item Point your browser to \href{https://tinyurl.com/QMCPyTutorial}{\nolinkurl{https://tinyurl.com/QMCPyTutorial}}
		\item Open the file in Google Colaboratory
		\item Make a copy of this file onto your own Google drive account \\
		\texttt{File} $\rightarrow$ \texttt{Save a copy in Drive}
	\end{itemize}
\end{frame}

\section{Why Collaborate?}

\begin{frame}
	\frametitle{Acknowledgments for QMCPy}
	
	\begin{itemize}
		\item Coded primarily by Aleksei Sorokin 
		
		\item Folded in code from several groups (see above)
		
		\item Funded and encouraged by Mike McCourt at SigOpt
	\end{itemize}
\end{frame}


\begin{frame}
	{The Guaranteed Automatic Integration Library (GAIL) and QMCPy Teams}
	
	\vspace{-2ex}
	\includegraphics[angle = 180, origin = c, width = 0.32\textwidth]{ProgramsImages/GAIL2014RE.jpeg} \
	\includegraphics[width = 0.32\textwidth]{ProgramsImages/GAILatSIAM2018Hi.jpeg} \ 
	\includegraphics[width = 0.32\textwidth]{ProgramsImages/GAILatChinatown2018.jpg}
	
	\vspace{-4ex}

	{\small 
		\hspace{-4ex}\begin{tabular}{p{0.545\textwidth}p{0.44\textwidth}}
		
		\begin{itemize}
			\setlength{\itemsep}{0ex}
	
			\item Sou-Cheng Choi (Chief Data Scientist, Kamakura)
			
			\item Yuhan Ding (IIT PhD '15, Lecturer, IIT)
			
			\item Lan Jiang  (IIT PhD '16, Compass)
			
			\item Llu\'is Antoni Jim\'enez Rugama (IIT PhD '17, UBS)
			
			\item Mike McCourt (IIT BS ''07, Cornell PhD '12, \\ Head of Research, SigOpt)
			
			\item Jagadeeswaran Rathinavel (IIT PhD '19, Wi-Tronix)
			
			
			
			
		\end{itemize}
		
		&
		
		\begin{itemize}
			
			\setlength{\itemsep}{0ex}
			
			\item Aleksei Sorokin (IIT BS \& MAS '21 exp.)
			
			\item  Xin Tong (IIT MS, UIC PhD '20 exp.)
			
			\item Kan Zhang (IIT PhD '20 exp.)
			
			\item Yizhi Zhang (IIT PhD '18, Jamran Int'l)
			
			\item Xuan Zhou (IIT PhD '15, JP Morgan)
			
			\item and others
			
			
		\end{itemize}
	
		
	\end{tabular}
}

\end{frame}

\begin{frame}{The Argument for Community Software}
	
	\vspace{-4ex}
	
	\begin{itemize}
	\item<1-> Our groups are typically expert at only parts of the whole picture:
			\begin{itemize}
			\item LD sequence generators
			
			\item Increasing efficiency, e.g., MLMC, MDM
			
			\item Stopping criteria
			
			\item Interesting use cases
		\end{itemize}
	
		A community library let's us take advantage of the best.

	\item<2-> We provide a consistent interface for the different pieces.
	
	\item<3-> Tedious stuff only needs to be done once.

		\item<4-> Many eyes help find and correct errors
		\begin{itemize}
			\item MATLAB's Sobol' generator's incorrect scrambling corrected in MATLAB 2017a after Tony Jim\'enez Rugama noticed discrepancies.
			
			\item PyTorch's Sobol' generator found to be wrong unless double precision is proactively specified; also missing the first point; reported at \url{https://github.com/pytorch/pytorch/issues/32047}.
		\end{itemize}
	

\end{itemize}
\end{frame}

\begin{frame}{How You Can Contribute}
	
		\vspace{-4ex}
	Try out  QMCPy and then
	
	\begin{description}
		\item[Easy] Submit your bugs and feature requests as issues to \url{https://github.com/QMCSoftware/QMCSoftware/issues}
		
		\item[Moderately Difficult]  Ask your students or collaborators to try QMCPy themselves and submit their bugs and feature requests
		
		\item[Heroic] Add a feature or use case and make a pull request at \url{https://github.com/QMCSoftware/QMCSoftware/pulls} \\ so that we can included it in our next release
	\end{description}

	
	Questions?  Email us at \href{mailto:qmc-software@googlegroups.com}{\nolinkurl{qmc-software@googlegroups.com}}
	
	\uncover<2>{And if you are wedded to another language, think of designing your software so that others can add to it easily.}

\end{frame}

\finalthanksnote{These slides are  available at \\  \href{https://speakerdeck.com/fjhickernell/quasi-monte-carlo-software}{\nolinkurl{speakerdeck.com/fjhickernell/quasi-monte-carlo-software}}\\
Google Colaboratory notebook at \href{https://tinyurl.com/QMCPyTutorial}{\nolinkurl{tinyurl.com/QMCPyTutorial}}\\
Blog at \href{https://qmcpy.wordpress.com/}{\nolinkurl{qmcpy.wordpress.com/}}}


\thankyouframe


\end{document}





