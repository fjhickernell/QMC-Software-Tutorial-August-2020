%
% sample root file for your "contribution" to a contributed volume
%
% Use this file as a template for your own input.
%
%%%%%%%%%%%%%%%% Springer %%%%%%%%%%%%%%%%%%%%%%%%%%%%%%%%%%
\documentclass[graybox,footinfo]{svmult}

\smartqed
\usepackage{mathptmx}       % selects Times Roman as basic font
\usepackage{helvet}         % selects Helvetica as sans-serif font
\usepackage{courier}        % selects Courier as typewriter font
\usepackage{type1cm}        % activate if the above 3 fonts are
% not available on your system
\usepackage{graphicx}       % standard LaTeX graphics tool
% when including figure files

\usepackage{array,colortbl}
\usepackage{amsmath,amsfonts,amssymb,bm} % no amsthm, Springer defines Theorem, Lemma, etc themselves
%\usepackage[mathx]{mathabx}
\DeclareFontFamily{U}{mathx}{\hyphenchar\font45}
\DeclareFontShape{U}{mathx}{m}{n}{
	<5> <6> <7> <8> <9> <10>
	<10.95> <12> <14.4> <17.28> <20.74> <24.88>
	mathx10
}{}
\DeclareSymbolFont{mathx}{U}{mathx}{m}{n}
\DeclareFontSubstitution{U}{mathx}{m}{n}
\DeclareMathAccent{\widecheck}      {0}{mathx}{"71}



% Note that Springer defines the following already:
%
% \D upright d for differential d
% \I upright i for imaginary unit
% \E upright e for exponential function
% \tens depicts tensors as sans serif upright
% \vec depicts vectors as boldface characters instead of the arrow accent
%
% Additionally we throw in the following common used macro's:
\input{macros}

% Macros below are now included in macros.tex from MCQMC 2016 web site
% This spot formerly included macros that are now in macros.tex

% indicator boldface 1:
\DeclareSymbolFont{bbold}{U}{bbold}{m}{n}
\DeclareSymbolFontAlphabet{\mathbbold}{bbold}
%\newcommand{\ind}{\mathbbold{1}}


\usepackage{microtype} % good font tricks

\usepackage[colorlinks=true,linkcolor=black,citecolor=black,urlcolor=black]{hyperref}
\urlstyle{same}
\usepackage{bookmark}
\pdfstringdefDisableCommands{\def\and{, }}
\makeatletter % to avoid hyperref warnings:
\providecommand*{\toclevel@author}{999}
\providecommand*{\toclevel@title}{0}
\makeatother



\usepackage{bbm,mathtools,array,longtable,booktabs,graphicx,color,enumitem}
%\input FJHDef.tex


\newcommand{\DHJRnorm}[2][{}]{\ensuremath{\left \lVert #2 \right \rVert}_{#1}}
\newcommand{\DHJRnormnorm}[2][{}]{\ensuremath{\lVert #2 \rVert}_{#1}}
\newcommand{\DHJRbignorm}[2][{}]{\ensuremath{\bigl \lVert #2 \bigr \rVert}_{#1}}
\newcommand{\DHJRBignorm}[2][{}]{\ensuremath{\Bigl \lVert #2 \Bigr \rVert}_{#1}}
\newcommand{\DHJRabs}[1]{\ensuremath{{\left \lvert #1 \right \rvert}}}
\newcommand{\DHJRbigabs}[1]{\ensuremath{{\bigl \lvert #1 \bigr \rvert}}}


\providecommand{\HickernellFJ}{Hickernell}

\definecolor{orange}{rgb}{1.0,0.3,0.0}
\definecolor{violet}{rgb}{0.75,0,1}
\newcommand{\frednote}[1]{  {\textcolor{red}  {\mbox{**Fred:} #1}}}
\newcommand{\yuhannote}[1]{ {\textcolor{violet}  {\mbox{**Yuhan:} #1}}}
\newcommand{\tonynote}[1]{ {\textcolor{orange}  {\mbox{**Tony:} #1}}}

%\journal{Journal of Complexity}

\allowdisplaybreaks[4]


\newcommand{\AGSComment}[1]{{\color{cyan} Aleksei: #1}}

\newcommand{\hmu}{\widehat{\mu}}

\begin{document}

\title*{Quasi-Monte Carlo Software}
\author{Sou-Cheng Terrya Choi \and Fred J. Hickernell \and Michael J. McCourt \and Aleksei Sorokin}
\institute{Sou-Cheng Terrya Choi \at Department of Applied Mathematics, Illinois Institute of Technology,\\ RE 220, 10 W.\ 32$^{\text{nd}}$ St., Chicago, IL 60616 \email{yding2@hawk.iit.edu}
\and
Fred J. Hickernell \at Center for Interdisciplinary Scientific Computation and \\
Department of Applied Mathematics, Illinois Institute of Technology \\ RE 220, 10 W.\ 32$^{\text{nd}}$ St., Chicago, IL 60616 \email{hickernell@iit.edu}
\and
Michael J. McCourt \at ???
\and 
Aleksei Sorokin \at
Department of Applied Mathematics, Illinois Institute of Technology,\\ RE 220, 10 W.\ 32$^{\text{nd}}$ St., Chicago, IL 60616 \email{asorokin@hawk.iit.edu}}

\maketitle

\abstract{This is the article based on my MCQMC 2020 Tutorial}



\section{Introduction}
Quasi-Monte Carlo (QMC) methods promise great efficiency gains over independent and identically distributed (IID) Monte Carlo (MC) methods.  In some cases QMC may achieve one hundredth of the error as IID MC in the same amount of time. Often, these efficiency gains are obtained simply by  replacing IID sampling by the low discrepancy (LD) sampling that is at the heart of QMC. 

If you are a practitioner, you would like to test whether QMC would speed your computation.  You would like easy access to the best QMC algorithms available.  If you are a theoretician or algorithm developer, you would like try out your best ideas on a variety of use cases to demonstrate their practical value.  

This tutorial points to some of the best QMC software available.  Moreover, we describe QMCPy \cite{QMCPy2020a}, which is designed to bedome a community owned Python library that combines the best QMC algorithms under a common user interface.  A demonstration of how QMCPy works is given in a Google colaboratory notebook \cite{QMCPyTutColab2020}.

The model problem for QMC is to approximate an integral:
\begin{equation} \label{eq:integral}
	\mu = \int_\Omega g(\bst) \lambda(\bst) \, \D \bst,
\end{equation}
where $g$ is the integrand, and $\lambda$ is a non-negative weight.  We use $\mu$ to denote the value of this integral because we perform a transformation to interpret it as the population mean of a random variable, $\mu = \bbE[f(\bsX)]$. A good choice can make computation more efficient.

 QMC approximates this population mean by a sample mean
\begin{equation} \label{eq:integral}
	\hmu = \frac 1n \sum_{i=1}^n f(\bsX_i), 
\end{equation}
where $\bsX_1, \bsX_2, \ldots$ is a carefully chosen sequence.  The choice of this sequence, and the choice of $n$ to satisfy  the prescribed error tolerance are important decisions.  QMC software helps the user make those decisions.

In the sections that follow we first overview available QMC software.  We next describe an architecture for good QMC software, i.e., what are the key components and how should they interact.  We then describe how we have implemented this architecture in QMCPy.  Finally, we summarize further directions that we hope QMCPy and related software projects will take.  Those interested in the following the development of QMCPy or even contributing to its growth are urged to visit the GitHub repository at \href{https://qmcsoftware.github.io/QMCSoftware/}{\nolinkurl{https://qmcsoftware.github.io/QMCSoftware/}}.

\section{Available Software for QMC}
QMC software spans three categories:  LD sequence generators, algorithms, and applications.  We review the better known software collections, recognizing that some software overlaps multiple categories.
Software focusing on generating high quality LD sequences  or their generators includes
\begin{description}[format=\textup,format=\textbf]
	\item[BRODA] Sobol' sequences in C, MATLAB, and Excel \cite{BRODA20a},
	\item[Burkhardt] various QMC software in C++, Fortran, MATLAB, \& Python \cite{Bur20a},
	\item[LatNet Builder] Generating vectors/matrices for lattices and digital nets \cite{LatNet},
	\item[MATLAB] Sobol' and Halton sequences, commercial \cite{MAT9.9},
	\item[MPS] Magic Point Shop, lattices and Sobol' sequences \cite{Nuy17a},
	\item[Owen] Randomized Halton sequences in R \cite{Owe20a},
	\item[PyTorch] Scrambled Sobol' sequences \cite{PyTorch},
	\item [qrng]  Sobol' and Halton sequences in R \cite{QRNG2020}, and
	\item[Robbe] LD Sequences in Julia \cite{Rob20a}.
\end{description}
Software focusing on QMC algorithms and applications includes
\begin{description}[format=\textup,format=\textbf]
	\item[GAIL] Automatic (Q)MC stopping criteria in MATLAB \cite{ChoEtal20a},
	\item[ML(Q)MC] Multi-Level (Quasi-)Monte Carlo routines in C++, MATLAB, Python, and R \cite{GilesSoft},
	\item[OpenTURNS] Open source initiative for the Treatment of Uncertainties, Risks 'N Statistics in Python \cite{OpenTURNS},
	\item[QMC4PDE] QMC for elliptic PDEs with random diffusion coefficients \cite{KuoNuy16a},
	\item[SSJ] Stochastic Simulation in Java \cite{SSJ}, and
	\item[UQLab] Framework for Uncertainty Quantification in MATLAB \cite{UQLab2014}.
\end{description}


\section{Components of QMC Software}

\section{Low Discrepancy Generators}

\subsection{Standard Uniform Distributions}

\subsection{General Distributions}

\section{Integration}

\subsection{Integrands}

\subsection{Stopping Criteria}

\section{Under the Hood}



\section{Further Work} \label{sec:further}


\begin{acknowledgement}
The authors would like to thank the organizers for a wonderful MCQMC 2018. 
We also thank the referees for their many helpful suggestions.  This work is supported in part by National Science Foundation grants DMS-1522687 and SigOpt.


\end{acknowledgement}

%\section*{References}
%\nocite{*}
\bibliographystyle{spmpsci.bst}
\bibliography{FJH23,FJHown23}


\end{document}

